\chapter*{Introduction}
\addcontentsline{toc}{chapter}{Introduction}

Durant notre master IVI\footnote{Le master Image Vision Interaction est 
une spécialité du master informatique de l'université de Lille 1}, 
nous avons l'occasion de réaliser un projet orienté recherche. Nous avons 
sélectionné le sujet \og Hand Kinect \fg, car nous nous intéressons à la 
modélisation 3D et aux dernières techniques d'acquisition.\\

L'homme utilise continuellement ses mains de manière naturelle. Nous 
sommes extrêmement précis et habiles avec elles. Elles nous permettent 
de ressentir des détails très fins. Au jour de l'écriture de ce rapport, 
utiliser les mains pour interagir avec un ordinateur reste encore limité.\\

% état de la question
Les études sont nombreuses, dans les domaines de la détection des mains 
et de leurs poses, ainsi que le suivi de celle-ci et la reconstruction 
3D des articulations des mains et de leurs formes. Cela fait plusieurs 
années, que la recherche scientifique et les techniques d'acquisition 
évoluent. Dans les années 90, la littérature considère principalement les 
caméras dont la sortie est RGB. Les méthodes proposent généralement la 
détection de la main, de sa pose et de sa forme. Ensuite, des techniques 
utilisant plusieurs caméras et des systèmes de marquage sont apparus. 
Enfin, depuis les années 2005, les caméras de profondeur ont permis 
l'amélioration et la création de certaines techniques. Maintenant, il 
existe des méthodes permettant de retrouver les articulations des mains 
et leurs reconstructions 3D. Les résultats de ces recherches se 
retrouvent dans les nouvelles IHM\footnote{Interface Homme-Machine}, 
afin de contrôler, interagir et communiquer avec une machine. Les enjeux 
sont d'autant plus intéressants actuellement, avec le développement des 
interfaces immersives.\\

% problèmatique
Dans ce domaine, les problématiques actuelles sont variées. Il est 
nécessaire de détecter les articulations de la main de manière précise. 
Les variations de luminosité sont courantes, la méthode doit être robuste 
à ces variations. L'application doit être temps réel, pour fonctionner de 
manière interactive.\\


% plan
Dans un premier temps, nous décrivons notre projet, afin de comprendre 
précisément le problème et les enjeux liés au projet. Ensuite, nous 
réalisons un état de l'art ainsi qu'un prévisionnel afin de déterminer 
les solutions adéquates pour notre problème et de planifier les tâches 
qui seront à réaliser pour la réussite du projet. Enfin, nous déterminons 
la meilleure solution pour notre projet.\\

