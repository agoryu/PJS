\section{Introduction}

Durant nos études en master informatique nous avons l'occasion de réaliser un projet
orienté recherche. Nous avons sélectionné le sujet \enquote{Hand Kinect}, car nous nous
intéressons à la modélisation 3D et aux nouvelles technologie de caméra.

\subsection{Présentation}
Dans cette première partie du projet, nous allons effectuer un état de la l'art sur des solutions
permettant de modéliser la main d'un individu et de localiser l'ensemble des articulations de celle-ci 
à partir d'une caméra nous fournissant une image RGB et une image contenant l'information de profondeur de la scène filmé.\\

De plus en plus d'application nécessite des IHM plus précisent et plus naturelles. Pour cela, la détection
de la main devient de plus en plus intéressant dans des simulations. Ce niveau de précision peut être
utile dans des simulations médicales où les mains sont l'outils principales de l'utilisateur.\\

\subsection{Contexte}
La réalisation de ce projet se fait avec l'équipe MIIRE\footnote{Multimedia, Image, Indexing and Recognition}. Cette équipe travaille sur la segmentation du corps, sur la détection et la
reconnaissance d'expression du visage, et sur l'acquisition de données 3D.
Le projet sur lequel nous travaillons est un projet que nous démarrons. Nous travaillons avec la caméra
Kinect2 et l'outils Unity3D pour la modélisation et l'animation de la main.

\subsection{Problèmatique}
La Kinect permet déjà de détecter un squelette contenant quinze représentant différentes articulation 
du corps humain. 
%a revoir
D'autre outils permettent de détecter les mains des utilisateurs, cependant leur
champ de vision n'est pas assez important pour diverse application.

\subsection{Objectif du projet}
L'objectif de notre projet est d'utiliser la Kinect2 dans le but de détecter les différentes articulations
de la main et de modéliser celle-ci dans une application Unity.
%TODO ajouter les différentes étape de nos recherches 