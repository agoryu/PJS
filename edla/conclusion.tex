\chapter*{Conclusion}
\addcontentsline{toc}{chapter}{Conclusion}
%solution envisagé
Durant cet état de l'art, nous avons étudié plusieurs publications portant sur la détection et le suivi
d'une main. Nous avons vu une première solution utilisant une image couleur. Cette solution était
intéressante quand les caméras de profondeur n'étaient pas très répandu. Au vu des résultats fourni dans
\cite{haarlike} cette méthode est efficace, mais elle n'est pas adaptée à la modélisation d'une main 3D, elle permet
de suivre la main et de reconnaître quelques postures. De plus, les outils que l'on nous fournit nous permettent
de gagner du temps sur certaine partie de notre développement et la méthode utilisant une image couleur
est assez compliqué à implémenter. C'est pourquoi nous avons choisi la seconde méthode qui est plus
simple à développer et plus précise grâce aux outils fournis par la Kinect. La difficulté de cette méthode
est l'adapation du modèle à la main de l'utilisateur.\\

%previsionnel
Nous avons donc prévu une première phase dans le projet où nous allons intégrer les outils et le modèle
dans une scène Unity3D au moyen de script. Cette première phase est partagée en plusieurs étapes permettant
de partager au mieux les tâches entre chacun d'entre nous. La seconde phase est la réalisation du projet en lui-même, c'est-à-dire 
que nous devons obtenir un modèle qui se déplace dans l'espace et effectue les mouvements de l'utilisateur.
La dernière phase est la réalisation d'une application de démonstration qui doit valoriser notre travail
en demandant à l'utilisateur d'effectuer une tâche qu'il ne pourrait réaliser avec d'autre périphérique comme
la souris et le clavier. Cette application doit montrer le niveau de précision qu'il est possible d'atteindre
avec ce genre de système.\\

%ce que nous avons appris
Cette première partie du projet a été très instructif que se soit sur l'apprentissage des techniques de détection et de manipulation
de modèle 3D que sur le plan de la collaboration en équipe. Nous avons énormément appris sur la rédaction d'un
état de l'art, ainsi que sur la compréhension d'article de recherche. Nous avons pu pleinement mettre à profit
nos cours sur \og l'initiation à la recherche et à l'innovation \fg. Suite à la réalisation de ce rapport nous avons pu
clairement définir les tâches et les méthodes à utiliser pour réaliser que ce projet soit un succès. 
