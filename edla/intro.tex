\section{Introduction}

Durant notre master IVI\footnote{Le master Image Vision Interaction est 
l'un des parcours des masters informatiques de l'université de Lille 1}, 
nous avons l'occasion de réaliser un projet orienté recherche. Nous avons 
sélectionné le sujet \og Hand Kinect \fg, car nous nous intéressons à la 
modélisation 3D et aux dernières techniques d'acquisition.

\subsection{Présentation}
Dans le cadre de ce projet, nous sommes amené à rédiger un état de la l'art.
Cette première partie, nous permet de synthétiser et sélectionner les 
solutions. Notre objectif étant de détecter la main et les articulations 
de la main d'une personne, afin de modéliser en 3D la main de cette personne, 
dans des plans de caméra global ou rapprocher. Cela doit être réalisable en 
temps réel et à partir d'une caméra nous fournissant une image RGB et une 
image contenant l'information de profondeur de la scène filmée.\\

De plus en plus d'application nécessite des IHM plus précisent et plus 
naturelles. Pour cela, la détection de la main devient de plus en plus 
intéressant pour les simulations. Ce niveau de précision peut être
utile dans des simulations médicales où les mains sont l'outils principales 
de l'utilisateur.\\ % Autres exemples?

\subsection{Contexte}
La réalisation de ce projet se fait avec l'équipe 3D SAM\footnote{Modeling 
and Analysis of Static and Dynamic Shapes}. Cette équipe conçoit de 
nouveaux outils et méthodes d'analyse des formes des objets 3D statiques 
et dynamique. Ils travaillent sur l'analyse de formes des objects 3D et la 
modélisation des variations des formes dans des vidéos 3D. Nous démarrons 
ce projet proposé par l'équipe, il ne s'agit pas d'une reprise d'un projet 
déjà entamé. Nous travaillons avec la caméra Kinect2 et l'outils Unity3D 
pour la modélisation et l'animation de la main.

\subsection{Problèmatique}
La Kinect permet déjà de détecter un squelette contenant quinze représentant 
différentes articulation du corps humain. 
%a revoir
D'autre outils permettent de détecter les mains des utilisateurs, cependant leur
champ de vision n'est pas assez important pour diverse application.

\subsection{Objectif du projet}
L'objectif de notre projet est d'utiliser la Kinect 2 dans le but de détecter les différentes articulations
de la main et de modéliser celle-ci dans une application Unity.
%TODO ajouter les différentes étape de nos recherches 