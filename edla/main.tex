\documentclass{article}

%\usepackage[latin1]{inputenc}
\usepackage[T1]{fontenc}
\usepackage[francais]{babel}
%\usepackage[T1]{fontenc}
%\usepackage{graphicx}
%\usepackage{apalike}
%\usepackage{setspace}
%\usepackage{cite}
\usepackage[utf8]{inputenc}

\usepackage{csquotes}

\title{État de l'art sur la détection, le suivi de la main et animation de celle-ci dans une scène 3D.}

\author{
Alexis Robache \\
\texttt{robache.alexis@gmail.com} \\
\and
Elliot Vanegue\\
\texttt{elliot.vanegue@gmail.com} \\
\and
Gaëtan Deflandre\\
\texttt{gaetan.deflandre@gmail.com} \\
}

%TODO modifier les référence pour que cela afficher des numéros -> citation beaucoup trop longue

\begin{document}
\maketitle
\begin{abstract}
%TODO
\end{abstract}

\section{Introduction}

Durant notre master IVI\footnote{Le master Image Vision Interaction est 
l'un des parcours des masters informatiques de l'université de Lille 1}, 
nous avons l'occasion de réaliser un projet orienté recherche. Nous avons 
sélectionné le sujet \og Hand Kinect \fg, car nous nous intéressons à la 
modélisation 3D et aux dernières techniques d'acquisition.



\section{Etat de l'art}

\subsection{Différents type de données}
Il existe différents types de périphériques permettant de récupérer des données.
Les types de données récoltés authorisent différents traitements.

Les données en RGB sont principalement utilisées pour 

\subsection{Modélisation de la main}
Pour mieux visualiser les actions réalisées par la main de l'utilisateur, il est nécessaire d'avoir
un modèle de la main qui soit réaliste et précis par rapport à la réalité. Pour cela, nous avons besoin d'un
mesh d'une main modèle. La méthode utilisé par \cite{export:217428} permet en utilisant l'algorithme
ICP\footnote{Iterated Closest Point} \cite{zhang:inria-00074899} de modifier le mesh de la main afin
que les points de celui-ci aient une distance moins importante avec le nuage de point fourni pas la 
caméra. La méthode de \cite{export:217428} permet également d'adapter le squelette du modèle de la 
main, ce qui permet d'avoir une précision plus importante lors de l'utilisation de l'application.
Cette méthode se repose sur le calcul de la fonction énergie.

%comprehension elliot
%Pour cette méthode il faut utiliser un model de base de la main, puis il faut récupérer un nuage de 
%point. Ensuite avec un systeme d'énergie il faut rapprocher les points du maillage de base avec les 
%point du nuage de point -> ICP-style

\subsection{Détection de la main}
La méthode de \cite{export:238453} permet de détecter la main de l'utilisateur à partir d'une
région d'intérêt autour de la main. Il est possible de ne pas avoir deux base d'apprendissage comme
il est proposé dans cette méthode, mais le SDK de la Kinect 2 afin de ne pas devoir prendre en 
compte la rotation de la main de l'utilisateur.
%comprehension elliot
% \begin{itemize}
% \item utilisation de données d'apprentissage pour déterminer des postures de main
% \item besoin d'une détermination global de la position et de l'orientation de la main -> kinect doit la fournir dans notre cas
% \item notion de golden energy -> a voir
% \item la golden energy permet d'avoir une plus grande précision sur le score permettant
% de déterminer la posture de la main
% \item a partir de la position global de la main -> creation d'une ROI raisonnable. Pour moi cette ROI est déjà
% calculé dans la librairie de kinect
% \item pour supprimer ce qui ne correspond pas à la main -> projection des vertex sur la depthmap? avec
% transformation de la distance 3D?
% \item la golden energy est calculé à partir de la depthmap et d'une image couleur? etrange au debut il dise qu'il
% utilise que la depthmap (dernier paragraphe de l'intro)
% \item deux couches d'apprentissage : une pour la rotation de la main et une seconde pour tout le reste comme
% la position des doigts
% \item distance de fonctionnement 0.5 - 4.5
% \end{itemize}

% Les questions pour l'équipe
% \begin{itemize}
% \item plus de précision sur la golden energy
% \item aura-t-on des données sur les postures de la main ou devra t on écrire un programme d'apprentissage
% \item le model de la va t elle contenir le squelette avec le skinning
% \end{itemize}



\subsection{Evaluation des solutions envisagées}

\section{Conclusion}
\bibliographystyle{ieeetr}
\bibliography{biblio}

\end{document}