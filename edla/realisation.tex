\chapter{Travaux réalisés}


\section{Séparation des tâche}
\paragraph{}
Nous constatons assez rapidement dans le projet qu'il est très difficile d'utiliser
le SDK de la Kinect et la DLL directement dans Unity. Plusieurs projets, trouvé durant nos recherche, utilisent un système 
de socket afin de pouvoir envoyer des informations à une application Unity3D à partir d'une application C\#.
Nous utilisons donc la même méthode afin de pouvoir séparer les différentes tâches de notre projet.

\paragraph{}
Nous créons donc un client utilisant la DLL et effectuant l'ensemble des opérations sur les données afin
de les envoyer à une application Unity qui n'a plus qu'à effectuer les transformations sur le modèle avec
les données reçu. En effet, les données fournis pas la DLL ne peuvent être appliqué tel quel sur le modèle.
Le repère des coordonnées fournis par la DLL est le repère image, donc un repère en 2D en pixel. Or, notre 
application doit modèliser la main dans un environnement 3D.

\begin{figure}[H]
  \begin{center}
    \includegraphics[width=350px]{images/schemaAppli.png}
    \caption{Schéma de la structure de l'application}
  \end{center}
\end{figure}

\paragraph{}
L'intérêt de cette séparation est que notre méthode de localisation des jointures se repose sur une 
DLL et que les calculs effectués sur les données sont spécifiques a celle-ci. En cas de changement de 
technologie, il suffit de modifier le client sans avoir à changer l'application Unity3D. La seul contrainte
étant la structure des socket envoyé. 

\section{Récupération de la position des jointure de la main}
\paragraph{}
Le premier problème à résoudre est donc le calcul de la troisième coordonnée de chaque jointure. Grâce à la Kinect,
il est possible de récupérer les coordonnées, dans le repère monde, de la jointure du poignet. Elle nous permet
également de récupérer une valeur dans l'image de profondeur, valeur représentant la troisième coordonné d'un point.
Nous avons alors la valeur du pixel de l'image de profondeur de la jointure recherché $p_{recherche}$, la valeur
du pixel de l'image de profondeur de la jointure du poignet $p_{ref}$ et la valeur de la troisième coordonné dans 
le repère monde de la jointure du poignet $z_{ref}$. Nous appliquons alors une règle de trois pour obtenir 
la valeur dans le repère monde de la troisième coordonné de la jointure courante.

\begin{equation}
 z_{recherche} = \frac{z_{ref} * p_{recherche}}{p_{ref}}
\end{equation}

\section{Modèlisation de la main}

\section{L'application de démonstration}