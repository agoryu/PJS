\chapter*{Introduction}
\addcontentsline{toc}{chapter}{Introduction}

Durant notre master IVI\footnote{Le master Image Vision Interaction est 
l'un des parcours des masters informatiques de l'université de Lille 1}, 
nous avons l'occasion de réaliser un projet orienté recherche. Nous avons 
sélectionné le sujet \og Hand Kinect \fg, car nous nous intéressons à la 
modélisation 3D et aux dernières techniques d'acquisition. \\

L'homme utilise continuellement ses mains de manière naturelle. Nous 
sommes extrêmement précis et habile avec elles. Elles nous permettent 
de resentir de très fins détails. Au jour de l'écriture de cette rapport,
utiliser les mains pour interagir avec un ordinateur reste encore limité.\\

% état de la question
Les études sont nombreuses, dans les domaines de la détection des mains 
et de leurs poses, ainsi que le suivit de celle-ci et la reconstruction 
3D des articulation des mains et de leurs formes. Cela fait plusieurs années, 
que la recherche scientique et les techniques d'acquisition évoluent. 
Dans les années 90, la littérature considère principalement les caméras 
dont la sorti est RGB. Les méthodes proposent généralement la détection 
de la main, de sa pose et de sa forme. Ensuite des techniques utilisant 
plusieurs caméras et un systèmes de marquage sont apparus. Enfin, depuis 
les années 2005, les caméras de profondeurs ont permis l'amélioration 
et la création de certaines techniques. Maintenant, il existe des méthodes 
accès permettant de retrouver les articulations des mains et leurs 
reconstructions 3D. Les résultats de ces recherches se retrouve dans les 
nouvelles IHM\footnote{Interactions Homme-Machine}, afin de contrôler, 
interaction et communiquer avec une machine. Les enjeux sont d'autant plus 
intéressant actuellement, avec le développement des interfaces 
immersives.\\

% problèmatique
Dans ce domaine, les problèmatiques actuelles sont variées. Il est 
nécessaire de détecter les acticulations de la main de manière précise. 
Les variations de luminausité sont courrante, il faut être robuste à 
ces variations. L'application doit être temps réel, pour fonctionné de 
manière interactif.\\


% plan
Dans un premier temps, nous décrivons notre projet, afin de comprendre 
précisement le problème et les enjeux liés au projet. Esuite, nous réalisons 
un état de l'art ainsi qu'un prévisionnel afin de déterminer les solutions 
adéquates pour notre problème et de plannifier les tâches quiseront à 
réaliser pour la réussite du projet. Enfin, nous déterminons la meilleur 
solution pour notre projet.



