\chapter*{Conclusion}
\addcontentsline{toc}{chapter}{Conclusion}
%solution envisagé
%TODO voir si on laisse cette parite et qu'on ajoute un truc concernant la dll ou si on supprime
Durant cet état de l'art, nous avons étudié plusieurs publications portant sur la détection et le suivi
d'une main. Nous avons vu une première solution utilisant une image couleur. Cette solution était
intéressante quand les caméras de profondeur n'étaient pas très répandues. Au vu des résultats fournis par S. Bilal
\cite{haarlike}, cette méthode est efficace, mais elle n'est pas adaptée à l'animation d'une main 3D. Elle permet seulement
de suivre la main et de reconnaître quelques postures. De plus, les outils que l'on nous fournit nous permettent
de gagner du temps sur certaines parties de notre développement et la méthode utilisant une image couleur
est assez compliquée à implémenter. C'est pourquoi nous avons choisi la seconde méthode qui est plus
simple à développer et plus précise grâce aux outils fournis par la Kinect 2. La difficulté de cette méthode
est l'adaptation du modèle à la main de l'utilisateur.\\

%previsionnel
% Nous avons donc prévu une première phase dans le projet où nous allons intégrer les outils et le modèle
% dans une scène Unity3D au moyen de script. Cette première phase est divisée en plusieurs étapes permettant
% de partager au mieux les tâches entre chacun d'entre nous. La seconde phase est la réalisation du projet en lui-même, c'est-à-dire 
% que nous devons obtenir un modèle qui se déplace dans l'espace et effectue les mouvements de l'utilisateur.
% La dernière phase est la réalisation d'une application de démonstration qui doit valoriser notre travail
% en demandant à l'utilisateur d'effectuer une tâche qu'il ne pourrait réaliser avec d'autres périphériques comme
% la souris et le clavier. Cette application doit montrer le niveau de précision qu'il est possible d'atteindre
% avec ce genre de système.\\

%ce que nous avons appris
%TODO modifier un peu pour que se soit d'actualité et ajout de l'apprentissage des outils mathematique et de développement + ajout 
%des notions vu en cours qui nous on aidé durant le projet
Cette première partie du projet a été très instructive, que ce soit sur l'apprentissage des techniques de détection et de manipulation
de modèle 3D, que sur le plan de la collaboration en équipe. Nous avons énormément appris sur la rédaction d'un
état de l'art, ainsi que sur la compréhension d'articles de recherche. Nous avons pu pleinement mettre à profit
nos cours sur \og l'initiation à la recherche et à l'innovation \fg. Suite à la réalisation de ce rapport, nous avons pu
clairement définir les tâches et les méthodes à utiliser pour que ce projet soit un succès. 
